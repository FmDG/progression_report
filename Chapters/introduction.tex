\section{Introduction}

\subsection{Background}

The intensification of Northern Hemisphere Glaciation (iNHG) is a major reorganisation of the Earth's climate and one of the largest secular changes in the last 5 million years \citep{driscollShortCircuitThermohaline1998, bartoliFinalClosurePanama2005}. The iNHG begun at the end of the Pliocene, roughly 2.7 million years ago (Ma), with the growth of large ice sheets across North America, Europe and Asia. The Pliocene was followed by the Pleistocene, characterised by the advance and retreat of ice sheets across the Northern Hemisphere, causing hundred-metre fluctuations in sea levels and marked climate changes. Permanent ice sheets were established on Greenland and West Antarctica over the iNHG, and atmospheric temperatures cooled considerably; 3 - 4\textdegree C in the tropics, over 10\textdegree C at the poles \citep{raymoInitiationNorthernHemisphere1994}. These changes also had a major impact on the internal structure of the world's oceans.

In the modern day, water in the deep oceans (below c. 3000 m depth) is produced by sinking water masses at high latitudes in the North Atlantic and Southern Oceans \citep{talleyClosureGlobalOverturning2013}. These water masses sink to depth, fill the deep oceans, and then upwell and recirculate. Deep waters formed in the North Atlantic upwell in the Southern Ocean, combine with surface waters and return to the North Atlantic, forming the Atlantic Meridional Overturning Circulation (AMOC) cell \citep{talleyChapterAtlanticOcean2011}. Despite similarly cold temperatures, deep waters are not formed in the modern North Pacific Ocean due to the presence of a strong salinity gradient, or halocline \citep{warrenWhyNoDeep1983}. The fresh surface waters of the subpolar North Pacific have a buoyancy that inhibits deep water formation. The deep Pacific Ocean is therefore filled with Antarctica Bottom Waters (AABW) and Circumpolar Deep Waters (CDW) from the Southern Ocean \citep{talleyClosureGlobalOverturning2013}.

Climate modelling results suggest that the North Pacific halocline was eroded in the Pliocene allowing for the formation of North Pacific Deep Waters (NPDW) and creating a Pacific Meridional Overturning Circulation (PMOC) cell \citep{burlsActivePacificMeridional2017}. This PMOC may have resulted from the reduced meridional sea surface temperature (SST) gradients and weaker atmospheric moisture transport in the Pliocene Pacific Ocean relative to today \citep{burlsActivePacificMeridional2017}. The evidence for deep water formation in the subpolar North Pacific is absent in the Early Pleistocene \citep{swannSalinityChangesNorth2010}, suggesting that the PMOC is halted over the iNHG \citep{burlsActivePacificMeridional2017}. This is most likely due to the establishment of a strong North Pacific halocline, but how and when this process occurred is poorly understood. The characteristics of the deep Pacific Ocean converge with those of the deep Atlantic over the iNHG \citep{woodardAntarcticRoleNorthern2014}, suggesting increased connectivity between the Atlantic and Pacific after the iNHG.

\subsubsection{The AMOC}

The strength of the AMOC over the iNHG is also uncertain, with contradictory evidence of both AMOC strengthening \citep{hayashiLatestPlioceneNorthern2020} and weakening \citep{langIncursionsSouthernsourcedWater2016} over the iNHG. Determining the change in AMOC strength is complicated by uncertainties over AMOC strength prior to the iNHG, in the Late Pliocene. A PMOC during the Late Pliocene may have weakened the AMOC in the Atlantic \citep{ferreiraAtlanticPacificAsymmetryDeep2018}, while climate modelling evidence suggests that AMOC strength remained similar to the modern in the Pliocene \citep{zhangMidplioceneAtlanticMeridional2013}. A strong AMOC would result in NADW export out of the Atlantic Ocean and could be responsible for the convergence in deep water mass properties between the Atlantic and Pacific Oceans over the iNHG \citep{kwiekPacificOceanIntermediate1999, woodardAntarcticRoleNorthern2014}.

\subsubsection{Impact of iNHG on Volcanism}

The iNHG is defined by the growth of major ice sheets across the Northern Hemisphere, and this expansion may have changed rates of volcanism on Iceland. Given the importance of volcanism in regulating the climate \citep{mckenzieContinentalArcVolcanism2016}, this changes in volcanism over the iNHG could have had wide-reaching climatic effects. At the end of the last glacial periods there is evidence for a marked increase in volcanism on Iceland as ice sheet retreated, reducing the ice loading on the crust and allowing more magma to rise to the surface and melt \citep{maclennanLinkVolcanismDeglaciation2002}. The iNHG would have caused rapid changes in the ice loading on Iceland as ice sheets advanced and retreated in the Early Pleistocene, which would have had an impact on the rates of volcanism on the island.

\subsection{Proposed Chapters, Knowledge Gaps and Aims}

The three chapters of this project will look at changes in: ocean structure in the Pacific, AMOC strength, and the rates of volcanism on Iceland in response to the iNHG.

\subsubsection{Ocean Structure in the Pacific}

The first chapter looks at ocean circulation in the Pacific over the iNHG. The circulation in the deep Pacific Ocean was likely different during the Pliocene compared to the present, as shown by the palaeo-proxy evidence from a PMOC \citep{burlsActivePacificMeridional2017,shanklePlioceneDecouplingEquatorial2021,fordSustainedMidPlioceneWarmth2022}. However, a PMOC is not simulated in most climate models \citep{tanModelingModernlikePCO22020a, zhangMidPlioceneAtlanticMeridional2021} which suggest that the mechanism for formation of an overturning cell in the Pacific is not full understood. The timing and nature of the cessation of the PMOC over the iNHG is not currently well resolved, and better constraining this process could shed light on the mechanism driving the basin-wide circulation.

There is evidence that points to a greater connectivity between the Atlantic and Pacific Oceans after the iNHG, with more heat and salt transport between the two basins \citep{woodardAntarcticRoleNorthern2014}. This increase in connectivity may be related to the shutdown of the PMOC over the iNHG or could be driven by changes in the Southern Ocean.

The aims of this first chapter are therefore:
\begin{enumerate}
	\item To characterise the temperature, carbon storage, and salinity of deep waters in the North Pacific.
	\item To understand how these properties change over the iNHG.
	\item To determine how the cessation of the PMOC fits in to the wider climate changes occurring over the iNHG. 
\end{enumerate}

\subsubsection{Changes in AMOC Strength}

The second chapter seeks to look at how the strength of the AMOC changed over the iNHG. There is contradictory evidence pointing to a stronger \citep{hayashiLatestPlioceneNorthern2020}, weaker \citep{langIncursionsSouthernsourcedWater2016}, and similar strength \citep{hodellHighResolutionStableIsotopic1992} AMOC to the present over the iNHG. The weaker AMOC signal after the iNHG may be the result of aliasing issues where weaker circulation during glacial periods of the Early Pleistocene decreases the average strength of Atlantic circulation \citep{raymoResponseDeepOcean1992}. AMOC strength is largely inferred from the spatial extent of North Atlantic Deep Water (NADW), with a larger extent implying stronger overturning \citep{hodellHighResolutionStableIsotopic1992,raymoResponseDeepOcean1992, langIncursionsSouthernsourcedWater2016}. However, modelling and palaeoenvironmental evidence suggest greater deep water formation in the Southern Ocean during the Late Pliocene \citep{hillModelledOceanChanges2017, mckayAntarcticSouthernOcean2012}, which would reduce the extent of northern-sourced deep waters without any change in overturning circulation strength.

The relative strength of deep-water export from the North Atlantic and Southern Ocean can be seen in the spatial extent of northern- and southern-sourced water masses in the deep mid-Atlantic Ocean. The characteristics of NADW and AABW are poorly defined over the iNHG in terms of temperature, salinity, and carbon storage. A clearer understanding of these properties could help to constrain past ocean circulation.

The aims of this chapter are:
\begin{enumerate}
	\item To describe the temperature, salinity, and carbon content of NADW and AABW water masses before and after the iNHG.
	\item To determine the relative influence of northern- and southern-sourced waters on the deep mid-Atlantic before and after the iNHG.
	\item To determine how AMOC strength changed over the iNHG and the role this may have played in wider oceanographic changes.
\end{enumerate}

\subsubsection{Impact on Volcanism}


The final chapter looks at the impacts that the iNHG may have had on volcanism. The iNHG likely caused a change in the rate of volcanism due to the loading of major ice sheets over areas such as Iceland. The impact of this process is poorly understood due to the limited number of records of volcanism from the Late Pliocene and Early Pleistocene. Tephra records from the Holocene and Late Pleistocene have shown that there are statistically significant changes in the rate of volcanism due to ice loading \citep{maclennanLinkVolcanismDeglaciation2002, sigmundssonClimateEffectsVolcanism2010, sigmundssonMultipleEffectsIce2012}. It is expected that there will be similar changes observed over the iNHG.

The aims of this final chapter are:
\begin{enumerate}
	\item To construct a record of volcanism on Iceland for the Late Pliocene and Early Pleistocene from tephra in marine sediments cores.
	\item To determine if there is a significant change in the rate of volcanism over the iNHG on Iceland.
\end{enumerate}

\subsection{Signficance}

The iNHG represents a uniquely accessible natural experiment of how the Earth’s climate responds to major climatic changes. The PMOC is a good example of the limits of climate models in explaining the behaviour of the climate in response to secular changes. As the world warms due to the effects of anthropogenic warming, it is increasingly important to understand the dynamics of ocean circulation and how they are influenced by rapid temperature changes.

The Pacific Ocean is a major store of carbon and heat in the modern oceans \citep{chengImprovedEstimatesOcean2017}. Changing the ventilation of the deep Pacific Ocean would have major impacts on the carbon balance, markedly increasing the amount of CO$_2$ in the atmosphere, as well as releasing heat that has been stored in the deep ocean. The deep oceans are also major stores of nutrients, the upwelling of which can profoundly impact coastal communities. The circulation of the deep Atlantic and Pacific Oceans therefore has major economic and ecological impacts. This project seeks to further our understanding of the deep circulation of these oceans, and how the deep sea carbon storage may have changed in the past.

Volcanism as well as the deep oceans can have large impacts on the climate in the short term \citep{rampinoClimateVolcanismFeedbackToba1993}. The interplay between volcanism and the climate on longer timescales is less well understood. This project would seek to address this, and therefore improve our knowledge of how the Earth works and how fundamental forces such as plate tectonics and the climate interact.

\subsection{Report Structure}

This report will undertake a thorough review on the existing literature on the topics mentioned in the introduction and then outline the path of the PhD project in addressing the aims mentioned above. The literature review is structured in two parts, the first dealing with oceanography and the second with volcanism. 

The oceanography section of the literature review begins by describing the changes that occurred over the iNHG and how well we can constrain the nature and timings of these events. Then it covers the modern structure of the Pacific, Atlantic and Southern Oceans. This is followed by a discussion of the Atlantic and Pacific Oceans over the Late Pliocene and the iNHG, focussing in particular on:

\begin{enumerate}
	\item The controversy over the extent of overturning circulation in the Late Pliocene Pacific Ocean.
	\item The possible explanations for the declining PMOC strength over the iNHG.
	\item The competing evidence for changes in AMOC strength over the iNHG.
	\item The reasons why we see a convergence in deep water mass properties in the Atlantic and Pacific Oceans over the iNHG.
\end{enumerate}

The final part of this section will cover the techniques that will be used to answer these questions: oxygen isotopes, Mg/Ca, B/Ca and $\varepsilon_{\text{Nd}}$. It will focus on how these methods will be used in this project and their limitations.

The volcanism section of the literature review begins with an outline of how ice sheets and volcanism interact during the recent past. It then looks at theories for how volcanism can be influenced by ice sheet volume, and how the iNHG could have influenced this. Finally, there is an examination of the methods that will be used, and the sites that we will be studying. 

The final section of the report will outline how these knowledge gaps will be answered. It will look at the results that have been obtained already from oxygen isotope and trace metal analysis in the North Pacific and the questions raised by these results. This section will then point to the techniques and samples that will be analysed to infer how AMOC strength changed over the iNHG. Finally, there will be a discussion on the methods I will use to determine if rates of volcanism changed over the iNHG, as well what contingencies I will put in place if there is not sufficient tephra preserved at the site.  





